
\documentclass[12pt]{article}
\usepackage[spanish,es-noshorthands]{babel}
\usepackage[utf8]{inputenc}
\usepackage[T1]{fontenc}
\usepackage{geometry}
\geometry{margin=1in}
\usepackage{hyperref}
\usepackage{array}
\usepackage{longtable}
\usepackage{tikz}
\usetikzlibrary{positioning,arrows.meta,shapes.multipart}
\usepackage{enumitem}
\usepackage{xcolor}

\title{\textbf{Documentación del Proyecto QuixoWeb}}
\author{BONILLA VASQUEZ BRANDON RAFAEL \\ NELSON RODRÍGUEZ LÓPEZ}
\date{\today}

\begin{document}
\maketitle

\section*{Integrantes y datos de Git}
\begin{longtable}{|p{4cm}|p{3cm}|p{4cm}|p{5cm}|}
\hline
\textbf{Nombre} & \textbf{Carné} & \textbf{Usuario de Git} & \textbf{Correo de Git} \\
\hline
BONILLA VASQUEZ BRANDON RAFAEL & FI17008598 & \verb|brandonbonillavasquez| & \verb|bbonilla40084@ufide.ac.cr| \\
\hline
NELSON RODRÍGUEZ LÓPEZ & FI20016869 & \verb|nelson2587| & \verb|nrodriguez70450@ufide.ac.cr| \\
\hline
\end{longtable}

\section*{Stack, frameworks y herramientas}
\begin{itemize}[leftmargin=*,nosep]
  \item \textbf{Framework web}: ASP.NET Core MVC (Multi-Page Application, MPA).
  \item \textbf{ORM}: Entity Framework Core (EF Core).
  \item \textbf{Motor de base de datos}: Microsoft SQL Server (puede usarse LocalDB durante desarrollo).
  \item \textbf{Lenguaje}: C\#  (controladores, servicios, modelos).
  \item \textbf{Cliente}: Vistas Razor + JavaScript (AJAX).
  \item \textbf{Herramientas}: .NET SDK, Visual Studio/VS Code, Git, LaTeX (PGF/TikZ) para este documento.
\end{itemize}

\section*{Tipo de aplicación}
\begin{itemize}[leftmargin=*,nosep]
  \item \textbf{MPA}: Se renderizan múltiples vistas Razor en servidor; además, el proyecto expone acciones que retornan JSON para actualizar la UI vía AJAX.
\end{itemize}

\section*{Arquitectura utilizada}
\begin{itemize}[leftmargin=*,nosep]
  \item \textbf{MVC}: Controladores (Home, Game, Stats) se utilizan para realizar peticiones y delegan en un servicio de dominio (\texttt{IGameService/\-GameService}). EF Core (\texttt{QuixoDbContext}) persiste entidades (\texttt{Partida}, \texttt{Jugada}, \texttt{Estadistica}). ViewModels (\texttt{GameViewModel}, \texttt{HistoryViewModel}, \texttt{StatsViewModel}) preparan datos para vistas/JSON.
\end{itemize}

\section*{Definición de la base de datos (PGF/TikZ)}
\begin{center}
\begin{tikzpicture}[
  table/.style={draw, rounded corners, thick, fill=blue!5, inner sep=4pt},
  >=Stealth
]

% Partidas
\node (partidas) [table, align=left, minimum width=8cm]
{\textbf{Partidas}\\\hrulefill\\
\begin{tabular}{@{}ll@{}}
PartidaId & INT (PK)\\
ModoJuego & TINYINT (2/4)\\
FechaHoraCreacion & DATETIME (default GETDATE())\\
TiempoTranscurrido & TIME\\
Estado & NVARCHAR(50) (\textit{EnCurso/Finalizada})\\
GanadorId & INT (NULL, modo 2: 1/2)\\
EquipoGanador & NVARCHAR(1) (NULL, modo 4: A/B)\\
\end{tabular}
};

% Jugadas
\node (jugadas) [table, right=3cm of partidas, align=left, minimum width=8cm]
{\textbf{Jugadas}\\\hrulefill\\
\begin{tabular}{@{}ll@{}}
JugadaId & INT (PK)\\
PartidaId & INT (FK \textrightarrow Partidas.PartidaId)\\
NumeroJugada & INT\\
JugadorActual & INT (1..4)\\
FilaOrigen/ColumnaOrigen & INT (0..4)\\
FilaDestino/ColumnaDestino & INT (0..4)\\
OrientacionPunto & TINYINT (NULL, 0..3)\\
EstadoTablero & NVARCHAR(MAX) (JSON)\\
TiempoTranscurrido & TIME\\
\end{tabular}
};

% Estadisticas
\node (estadisticas) [table, below=2.0cm of partidas, align=left, minimum width=8cm]
{\textbf{Estadisticas}\\\hrulefill\\
\begin{tabular}{@{}ll@{}}
EstadisticaId & INT (PK)\\
ModoJuego & TINYINT (2/4)\\
JugadorEquipo & NVARCHAR(20) (\textit{Jugador1/Jugador2/EquipoA/EquipoB})\\
PartidasGanadas & INT\\
PartidasJugadas & INT\\
\end{tabular}
};

% Relaciones
\draw[->, thick] (jugadas.west) -- node[midway, above]{FK} (partidas.east);

\end{tikzpicture}
\end{center}

\section*{Referencias y recursos consultados}
\begin{itemize}[leftmargin=*,nosep]
  \item ASP.NET Core MVC (Controllers y Views): \href{https://learn.microsoft.com/aspnet/core/mvc/overview}{learn.microsoft.com/aspnet/core/mvc}
  \item Entity Framework Core (Modelos, relaciones e índices): \href{https://learn.microsoft.com/ef/core/}{learn.microsoft.com/ef/core}
  \item System.Text.Json (serialización a JSON): \href{https://learn.microsoft.com/dotnet/standard/serialization/system-text-json-overview}{learn.microsoft.com/dotnet/standard/serialization/system-text-json-overview}
  \item ASP.NET Core Session: \href{https://learn.microsoft.com/aspnet/core/fundamentals/app-state}{learn.microsoft.com/aspnet/core/fundamentals/app-state}
  \item SQL Server GETDATE(): \href{https://learn.microsoft.com/sql/t-sql/functions/getdate-transact-sql}{learn.microsoft.com/sql/t-sql/functions/getdate-transact-sql}
  \item LaTeX PGF/TikZ: \href{https://ctan.org/pkg/pgf}{ctan.org/pkg/pgf}
\end{itemize}

\section*{Prompts y agentes de IA usados}
\begin{itemize}[leftmargin=*,nosep]
  \item Interacción con M365 Copilot para: explicación de código, diseño de documentación, y generación de diagramas y secciones del README.
  \item Incluya aquí los \textit{prompts} exactos de entrada y los fragmentos de salida que se utilizaron durante el desarrollo y documentación.
   \item Interacción con ChatGPT (GPT-5.1 Thinking) para la creación del código, resolución de errores y documentación técnica del proyecto.
   \begin{itemize}[leftmargin=*,nosep]
      \item \href{https://chatgpt.com/g/g-p-68db32af14cc81919ae9ec4cd8b77bb8-quixo-progra-web/project}{Repositorio y desarrollo general del proyecto Quixo Web}.
      \item \href{https://chatgpt.com/g/g-p-68ef1ba5c78c8191bdacdd26f6d05d06-deberes-quixo/project}{Sesiones de apoyo para la documentación y tareas complementarias}.
      \item \href{https://chatgpt.com/g/g-p-68ef1ba5c78c8191bdacdd26f6d05d06/c/68ef0b43-e494-832e-bd99-6eba368fe173}{Generación del README en LaTeX y explicación detallada del código}.
      \item \href{https://chatgpt.com/g/g-p-68db32af14cc81919ae9ec4cd8b77bb8-quixo-progra-web/c/68dc984b-a66c-8321-b3dd-01e5d6c52133}{Discusión de funcionalidades específicas y pruebas}.
      \item \href{https://chatgpt.com/g/g-p-68db32af14cc81919ae9ec4cd8b77bb8/c/68db2e32-1584-832b-94e0-f2eb85701e5c}{Ajustes finales y preparación para la exposición}.
   \end{itemize}
   \href{https://ufidelitas-my.sharepoint.com/:b:/r/personal/nrodriguez70450_ufide_ac_cr/Documents/Documentos/Progra-web/%F0%9F%A7%A9%20Proyecto%20Quixo%20Web.pdf?csf=1&web=1&e=xg9JB6}
   \href {https://github.com/BrandonBonillaVasquez/FI17008598.git}
\end{itemize}

\section*{Instructivo}
\subsection*{Prerrequisitos}
\begin{itemize}[leftmargin=*,nosep]
  \item .NET SDK 7/8 instalado (\href{https://dotnet.microsoft.com/download}{dotnet.microsoft.com/download}).
  \item SQL Server (LocalDB durante desarrollo) y una cadena de conexión \texttt{QuixoConnection} en \texttt{appsettings.json}.
\end{itemize}

\subsection*{Instalación}
\begin{enumerate}[leftmargin=*,nosep]
  \item Clonar el repositorio: \verb|git clone https://github.com/BrandonBonillaVasquez/FI17008598.git|
  \item Configurar la cadena de conexión en \texttt{appsettings.json}, por ejemplo:
\begin{verbatim}
{
  "ConnectionStrings": {
    "QuixoConnection": "Server=(localdb)\\MSSQLLocalDB;Database=QuixoDb;Trusted_Connection=True;MultipleActiveResultSets=true"
  }
}
\end{verbatim}
  \item Restaurar paquetes: \verb|dotnet restore|
  \item Aplicar migraciones y crear/actualizar la base: \verb|dotnet ef database update|
\end{enumerate}

\subsection*{Compilación / Creación}
\begin{verbatim}
dotnet build
\end{verbatim}

\subsection*{Ejecución}
\begin{verbatim}
dotnet run
\end{verbatim}
La aplicación expondrá rutas como \verb|/Home/Index|, \verb|/Game/SelectMode|, \verb|/Stats/Index|.

\subsection*{Compilar este README en PDF (LaTeX)}
\begin{enumerate}[leftmargin=*,nosep]
  \item Compilar localmente: \verb|pdflatex README.tex| (ejecutar 2 veces si es necesario).
  \item Alternativa: Importar \texttt{README.tex} en Overleaf y exportar \texttt{README.pdf}.
\end{enumerate}

\section*{Notas finales}
\begin{itemize}[leftmargin=*,nosep]
  \item El repositorio \textbf{debe incluir todo el código fuente} y \textbf{excluir artefactos compilados} (carpetas \texttt{bin/} y \texttt{obj/}).
  \item Incluya \texttt{README.tex} y el \texttt{README.pdf} generado con LaTeX en la raíz del proyecto.
\end{itemize}

\end{document}